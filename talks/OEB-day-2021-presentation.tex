% Options for packages loaded elsewhere
\PassOptionsToPackage{unicode}{hyperref}
\PassOptionsToPackage{hyphens}{url}
%
\documentclass[
  ignorenonframetext,
]{beamer}
\usepackage{pgfpages}
\setbeamertemplate{caption}[numbered]
\setbeamertemplate{caption label separator}{: }
\setbeamercolor{caption name}{fg=normal text.fg}
\beamertemplatenavigationsymbolsempty
% Prevent slide breaks in the middle of a paragraph
\widowpenalties 1 10000
\raggedbottom
\setbeamertemplate{part page}{
  \centering
  \begin{beamercolorbox}[sep=16pt,center]{part title}
    \usebeamerfont{part title}\insertpart\par
  \end{beamercolorbox}
}
\setbeamertemplate{section page}{
  \centering
  \begin{beamercolorbox}[sep=12pt,center]{part title}
    \usebeamerfont{section title}\insertsection\par
  \end{beamercolorbox}
}
\setbeamertemplate{subsection page}{
  \centering
  \begin{beamercolorbox}[sep=8pt,center]{part title}
    \usebeamerfont{subsection title}\insertsubsection\par
  \end{beamercolorbox}
}
\AtBeginPart{
  \frame{\partpage}
}
\AtBeginSection{
  \ifbibliography
  \else
    \frame{\sectionpage}
  \fi
}
\AtBeginSubsection{
  \frame{\subsectionpage}
}
\usepackage{amsmath,amssymb}
\usepackage{lmodern}
\usepackage{ifxetex,ifluatex}
\ifnum 0\ifxetex 1\fi\ifluatex 1\fi=0 % if pdftex
  \usepackage[T1]{fontenc}
  \usepackage[utf8]{inputenc}
  \usepackage{textcomp} % provide euro and other symbols
\else % if luatex or xetex
  \usepackage{unicode-math}
  \defaultfontfeatures{Scale=MatchLowercase}
  \defaultfontfeatures[\rmfamily]{Ligatures=TeX,Scale=1}
\fi
% Use upquote if available, for straight quotes in verbatim environments
\IfFileExists{upquote.sty}{\usepackage{upquote}}{}
\IfFileExists{microtype.sty}{% use microtype if available
  \usepackage[]{microtype}
  \UseMicrotypeSet[protrusion]{basicmath} % disable protrusion for tt fonts
}{}
\makeatletter
\@ifundefined{KOMAClassName}{% if non-KOMA class
  \IfFileExists{parskip.sty}{%
    \usepackage{parskip}
  }{% else
    \setlength{\parindent}{0pt}
    \setlength{\parskip}{6pt plus 2pt minus 1pt}}
}{% if KOMA class
  \KOMAoptions{parskip=half}}
\makeatother
\usepackage{xcolor}
\IfFileExists{xurl.sty}{\usepackage{xurl}}{} % add URL line breaks if available
\IfFileExists{bookmark.sty}{\usepackage{bookmark}}{\usepackage{hyperref}}
\hypersetup{
  pdftitle={FBES/OEB/SenPEP weather station},
  pdfauthor={Pedro J. Aphalo (SenPEP)},
  hidelinks,
  pdfcreator={LaTeX via pandoc}}
\urlstyle{same} % disable monospaced font for URLs
\newif\ifbibliography
\usepackage{longtable,booktabs,array}
\usepackage{calc} % for calculating minipage widths
\usepackage{caption}
% Make caption package work with longtable
\makeatletter
\def\fnum@table{\tablename~\thetable}
\makeatother
\usepackage{graphicx}
\makeatletter
\def\maxwidth{\ifdim\Gin@nat@width>\linewidth\linewidth\else\Gin@nat@width\fi}
\def\maxheight{\ifdim\Gin@nat@height>\textheight\textheight\else\Gin@nat@height\fi}
\makeatother
% Scale images if necessary, so that they will not overflow the page
% margins by default, and it is still possible to overwrite the defaults
% using explicit options in \includegraphics[width, height, ...]{}
\setkeys{Gin}{width=\maxwidth,height=\maxheight,keepaspectratio}
% Set default figure placement to htbp
\makeatletter
\def\fps@figure{htbp}
\makeatother
\setlength{\emergencystretch}{3em} % prevent overfull lines
\providecommand{\tightlist}{%
  \setlength{\itemsep}{0pt}\setlength{\parskip}{0pt}}
\setcounter{secnumdepth}{-\maxdimen} % remove section numbering
\ifluatex
  \usepackage{selnolig}  % disable illegal ligatures
\fi

\title{FBES/OEB/SenPEP weather station}
\subtitle{Viikki, Helsinki, Finland}
\author{Pedro J. Aphalo (SenPEP)}
\date{2021-03-08}

\begin{document}
\frame{\titlepage}

\begin{frame}{Our weather station}
\protect\hypertarget{our-weather-station}{}
\begin{figure}
\centering
\includegraphics{images/tower-2020-11-small.jpg}
\caption{\textbf{Figure} View of the station after the infrared
temperature sensors were installed in November 2020 on a third
cross-arm.}
\end{figure}
\end{frame}

\begin{frame}{What makes this station different?}
\protect\hypertarget{what-makes-this-station-different}{}
\begin{itemize}
\tightlist
\item
  Time interval for acquisition of measurements: 5 s.
\item
  Time interval for logging of summaries: 1 min, 1 h, 1 d.
\item
  Solar radiation measurement: currently seven types of sensors.
\item
  Soil measurements as a depth profile.
\item
  For research at the experimental field of the Viikki campus it
  provides on-site data.
\end{itemize}
\end{frame}

\begin{frame}{Example: photosynthetically active radiation}
\protect\hypertarget{example-photosynthetically-active-radiation}{}
Data for one day in the summer, cloudy morning, sunny afternoon.

\includegraphics{OEB-day-2021-presentation_files/figure-beamer/unnamed-chunk-5-1.pdf}
\end{frame}

\begin{frame}{Example: photosynthetically active radiation}
\protect\hypertarget{example-photosynthetically-active-radiation-1}{}
Monthly density distributions of 1 minute average photon irradiances for
the sun above the horizon. Total number of observations = 900542.

\includegraphics{OEB-day-2021-presentation_files/figure-beamer/unnamed-chunk-6-1.pdf}
\end{frame}

\begin{frame}{Air and surface temperatures}
\protect\hypertarget{air-and-surface-temperatures}{}
Same day as above, cloudy morning and sunny afternoon.

\includegraphics{OEB-day-2021-presentation_files/figure-beamer/unnamed-chunk-7-1.pdf}
\end{frame}

\begin{frame}{Soil temperature}
\protect\hypertarget{soil-temperature}{}
\includegraphics{OEB-day-2021-presentation_files/figure-beamer/unnamed-chunk-8-1.pdf}
\end{frame}

\begin{frame}{Profile of soil temperature}
\protect\hypertarget{profile-of-soil-temperature}{}
Here we fit a quantile regression for soil temperature on soil depth for
the 5\%, 50\% and 95\% percentiles. This highlights the range of
variation at different depths on different months of the year. The data
values, not plotted, are medians from three sensors.

\includegraphics{OEB-day-2021-presentation_files/figure-beamer/unnamed-chunk-9-1.pdf}
\end{frame}

\begin{frame}{Limitations of the data}
\protect\hypertarget{limitations-of-the-data}{}
The intention when setting up the staion was to \emph{take readings
during the growing season}. With the current instrumentation:

\begin{itemize}
\tightlist
\item
  Radiation data in winter months is unreliable as the sensors lack
  blowers to clear the snow.
\item
  Precipitation as snow or sleet is not detected by the weather sensor.
\item
  Wind measurements can be potentially affected by snow and ice
  accumulation.
\item
  Other variables can be trusted year-round.
\end{itemize}

If year-round data are needed, update would be possible with a moderate
investement.
\end{frame}

\begin{frame}{Logger and sensors}
\protect\hypertarget{logger-and-sensors}{}
\begin{itemize}
\tightlist
\item
  Campbell Scientific CR6 datalogger with a CDM-A116 module
\item
  The datalogger has both digital and analogue inputs.
\item
  Analogue to digital conversion (ADC) is done with very high
  resolution.
\item
  Data are downloaded on site through a USB connection to a laptop
  computer.
\item
  The logger is powered from mains but has a solar panel and battery as
  backup.
\end{itemize}
\end{frame}

\begin{frame}{Solar radiation sensors}
\protect\hypertarget{solar-radiation-sensors}{}
\begin{longtable}[]{@{}llll@{}}
\toprule
Sensor type & make & variable & qty. \\
\midrule
\endhead
UV-Cosine (UVB) & sglux & UVB irradiance & 1 \\
UV-Cosine (UVA) & sglux & UVA irradiance & 1 \\
UV-Cosine (blue) & sglux & Blue-Violet irrad. & 1 \\
SKR-110 R/FR & Skye & red irradiance & 1 \\
SKR-110 R/FR & Skye & far-red irradiance & 1 \\
LI-190 quantum & LI-COR & PAR (total PPFD) & 1 \\
CS310 & CampbellSci & PAR (total PPFD) & 1 \\
SMP3 pyramometer & Kipp & global radiation & 1 \\
BF5 & Delta-T & PAR (total PPFD) & 1 \\
BF5 & Delta-T & PAR (diffuse PPFD) & 1 \\
\bottomrule
\end{longtable}
\end{frame}

\begin{frame}{Weather sensors}
\protect\hypertarget{weather-sensors}{}
\begin{longtable}[]{@{}llll@{}}
\toprule
Sensor type & make & variable & qty. \\
\midrule
\endhead
WXT520 & Vaisala & Air temperature & 1 \\
WXT520 & Vaisala & Air humidity & 1 \\
WXT520 & Vaisala & Wind speed & 1 \\
WXT520 & Vaisala & Wind direction & 1 \\
WXT520 & Vaisala & Atmospheric pressure & 1 \\
WXT520 & Vaisala & Precipitation, rain & 1 \\
WXT520 & Vaisala & Precipitation, hail & 1 \\
CSmicro LT02 & Optris & Surface temperature & 2 \\
\bottomrule
\end{longtable}
\end{frame}

\begin{frame}{Soil sensors (continued)}
\protect\hypertarget{soil-sensors-continued}{}
\begin{longtable}[]{@{}llll@{}}
\toprule
Sensor type & make & variable & qty. \\
\midrule
\endhead
SoilVUE10 & CampbellSci & Soil moisture profile & 3 \\
SoilVUE10 & CampbellSci & Soil temperature profile & 3 \\
SoilVUE10 & CampbellSci & Soil elect. cond. profile & 3 \\
SoilVUE10 & CampbellSci & Soil permittivity profile & 3 \\
CS655 & CampbellSci & Soil moisture & 8 \\
CS655 & CampbellSci & Soil temperature & 8 \\
CS655 & CampbellSci & Soil elect. cond. & 8 \\
CS655 & CampbellSci & Soil permittivity & 8 \\
107 & CampbellSci & Soil temperature profile & 3 \\
\bottomrule
\end{longtable}
\end{frame}

\begin{frame}{Quantities logged once per minute}
\protect\hypertarget{quantities-logged-once-per-minute}{}
\begin{longtable}[]{@{}lll@{}}
\toprule
Quantity & Summary & Units \\
\midrule
\endhead
PAR total (sensor 1) & mean of 12 values & µmol/s/m² \\
PAR total (sensor 2) & mean of 12 values & µmol/s/m² \\
PAR diffuse (sensor 2) & mean of 12 values & µmol/s/m² \\
Global radiation & mean of 12 values & W/m² \\
Red light & mean of 12 values & µmol/s/m² \\
Far-red light & mean of 12 values & µmol/s/m² \\
Red:fra-red ratio & mean of 12 values & mol / mol \\
Blue light & mean of 12 values & µmol/s/m² \\
UV-A radiation & mean of 12 values & µmol/s/m² \\
UV-B radiation & mean of 12 values & µmol/s/m² \\
\bottomrule
\end{longtable}
\end{frame}

\begin{frame}{Quantities logged once per minute (continued)}
\protect\hypertarget{quantities-logged-once-per-minute-continued}{}
\begin{longtable}[]{@{}lll@{}}
\toprule
Quantity & Summary & Units \\
\midrule
\endhead
Wind speed & sample & m/s \\
Wind direction & sample & Deg \\
Air temperature & sample & C \\
Air relative humidity & sample & \% \\
Air dew point & sample & C \\
Atmospheric pressure & sample & hPa \\
Rain & sample & mm \\
Hail & sample & hits/cm2 \\
Ground surface temperature & mean of 12 values & C \\
Vegetation surface temperature & mean of 12 values & C \\
Soil temperature at approx. 3 cm & mean of 12 values & C \\
\bottomrule
\end{longtable}
\end{frame}

\begin{frame}{Quantities added based on the time stamp}
\protect\hypertarget{quantities-added-based-on-the-time-stamp}{}
Due to summer/winter time and small shifts in the logger's clock the
logged time if first adjusted to UTC when needed.

\begin{longtable}[]{@{}lll@{}}
\toprule
Quantity & Summary & Units \\
\midrule
\endhead
Time & sample & yyyy-mm-dd hh:mm:ss UTC \\
Day of year & sample & numeric \\
Month of year & sample & 1..12 \\
Month name & sample & character \\
Calendar year & sample & numeric \\
Tim of day & sample & numeric \\
Solar time & sample & numeric \\
Sun elevation & sample & degrees \\
Sun azimuth & sample & degrees \\
\bottomrule
\end{longtable}
\end{frame}

\begin{frame}{Quantities logged once per hour}
\protect\hypertarget{quantities-logged-once-per-hour}{}
In addition to mean and standard deviations for the same variables as
logged each minute.

\begin{longtable}[]{@{}lll@{}}
\toprule
Quantity & Summary & Units \\
\midrule
\endhead
Soil water content & sampled at 5, 10, 20, 30, 40 and 50 cm depth &
m3/m3 \\
Electric conductivity & sampled at 5, 10, 20, 30, 40 and 50 cm depth
& \\
Soil temperature & sampled at 5, 10, 20, 30, 40 and 50 cm depth & C \\
Soil water content & sampled at one depth & m3/m3 \\
Electric conductivity & sampled at one depth & \\
Soil temperature & sampled at one depth & C \\
\bottomrule
\end{longtable}
\end{frame}

\begin{frame}{Quantities logged once per day}
\protect\hypertarget{quantities-logged-once-per-day}{}
In addition to daily maxima and minima and the times of occurence for
most variables mentioned above, daily histograms are recorded.

\begin{longtable}[]{@{}lll@{}}
\toprule
Quantity & Summary & Units \\
\midrule
\endhead
PAR & histogram with 25 bins & frq. µmol/s/m² \\
Global radiation & histogram with 25 bins & frq. W/m² \\
PAR (log scaled) & histogram with 12 bins & frq. log(µmol/s/m²) \\
Global radiation (log scaled) & histogram with 10 bins & frq.
log(W/m²) \\
\bottomrule
\end{longtable}
\end{frame}

\begin{frame}{Data availability}
\protect\hypertarget{data-availability}{}
Available on request from Pedro J. Aphalo. Online access planned.
Acknowledgment of source required, collaborations are also welcome. The
weather station is registered in TUHAT as infrastructure under the name
\href{https://tuhat.helsinki.fi/admin/editor/dk/atira/pure/api/shared/model/equipment/editor/equipmenteditor.xhtml?id=159814768}{FBES/OEB/SenPEP
meteorological station}, if you use or have used any data from the
station in a publication or activity please, link the corresponding
TUHAT entries to this entry.
\end{frame}

\end{document}
